\documentclass[12pt]{article} 
\usepackage{epsfig,graphicx,color,url,verbatim}
\usepackage{geometry}
\usepackage{color}
\usepackage{subfig}
\usepackage{amsmath}
\usepackage[countmax]{subfloat}
\usepackage{parskip}
\setlength{\parindent}{.25in}
\captionsetup{justification=centering}
\usepackage{setspace}
\onehalfspacing
\geometry{margin=1.0in}

\title{{\bf Flatworld Conquest:\\
    Special Agents}}

\author{Mike Jacobi, Nick Aase\\
University of New Mexico: Dept. of Computer Science\\
\normalsize  mikerjacobi@gmail.com \& aase@unm.edu}

\date{}

\begin{document}
\begin{sloppypar}

\maketitle


%%%%%%%%%%%%%%%%%%%%%%%%%%%%%%%%%%%%%%%%%%%%%%%%%%%%%%%%%%%%%%%%%%%%%%%%%%%
%% Reminders:
%% -Every figure and table MUST have a numbered caption that defines all 
%% symbolism, and reference any sources.
%%
%% -Number every equation to the justified right and define every symbol in 
%% the paragraph immediately after the equation without indentation.\\
%%
%% Also, I tend to manually terminate my lines to somewhere less than 80
%% columns. You don't have to, but it doesn't change the final appearance
%% of the pdf, I just find it easier to read.
%%%%%%%%%%%%%%%%%%%%%%%%%%%%%%%%%%%%%%%%%%%%%%%%%%%%%%%%%%%%%%%%%%%%%%%%%%%

\renewcommand{\abstractname}{\Large Abstract}
\begin{abstract}
  \begin{singlespace}
    Within the training world Flatworld, we set out to create a series of 
    progressively more effective and complicated ``brains'' for an agent.
    The goal was to keep it alive as long as possible by having it act 
    intelligently within Flatworld. We implemented these brains by developing 
    a series of interlaced neural networks which governed the agent's actions.

    Through generational improvement of the agent's senses and functions,
    we were able to go from a vegetative agent who gradually starved to death,
    to one who actively sought out food and avoided poisonous objects. The
    final model of our agent's brain gave it a lifespan approximately 
    fifteen times longer than its idle ancestor.
  \end{singlespace}
\end{abstract}

\section{Introduction} \label{sec:Introduction}

(all major section headings 14 point, bold, numbered.  All others are 12 point, underlined.)

we cite shit via\cite{Anders}

\begin{itemize}
  \item
    start broad then focus
  \item
    what is the general area you are studying 
  \item
    what is the specific area 
  \item
    what is the problem you are addressing
  \item
    what have others done (references) (NA: shit... references? what the
    hell references do we need? Eeep!! The bibliography section is in
    \texttt{main.tex}).
  \item
    why is this not enough or incomplete
  \item
    what follows in this paper
\end{itemize}

\subsection{This is a subection}
foo.

\section{Approach} \label{sec:Approach}

%%%%%%%%%%%%%%%%%%%%%%%%%%%%%%%%%%%%%%%%%%%%%%%%%%%%%%%%%%%%%%%%%%%%%%%%%%%
%% Discuss:
%%     your approach to the problem
%%   
%%     why its good or better
%%   
%%     details of solution (no results)
%%   
%%     where did your data come from
%%   
%%     where did your model(s) come from
%%   
%%     algorithms to be used
%%   
%%     how will you explore the parameter space
%% 
%%     analysis to be done on data
%% 
%%%%%%%%%%%%%%%%%%%%%%%%%%%%%%%%%%%%%%%%%%%%%%%%%%%%%%%%%%%%%%%%%%%%%%%%%%%
In this section we will first present reasons \emph{why} the afforementioned
programming methods will not work well, followed by motivation to use the
methods we did. After these rather nebulous postulations, we will discuss
\emph{our} particular implementation, starting from the ground up. Finally
we will explain what data we collected and why we believed they were important.

\subsection{Ineffective Theory}

\subsubsection{Traditional Procedural Programming}
Consider a stream or river. (Disclaimer: this is a massive simplification of
one of natures most majestic features.) Despite potentially massive deluges, 
the model is still fairly predictible: water starts at the source and flows 
down to the mouth. If we use the analogy of the water in a river being data 
flow in a program, we can say that if the water reaches the river delta, it
has sucessfully arrived at its goal.

But what of lakes and seas? There is still current in the water, but when
can one say that it has reached its ``goal?'' Where did it start in the first
place?

Likewise, when has our agent reached its goal? Perhaps we could say it is 
when the agent has consumed all the food in Flatworld, but was the agent 
efficient in doing so? Let us consider the potential program flow that
would be required for modelling an agent's simple decision whether to eat one
object it sees compared with another:

\begin{verbatim}
  if object0 is food
    if object0 is only object in sight
      eat object0
    else if object1 is food
      if object1 is closer than object0
        eat object1 
      else
        eat object0
  rinse and repeat
\end{verbatim}

This could be rearranged in multiple ways to produce the same behavior,
but the result would be the same; even this simple action is quite cumbersome
to implement.

\subsubsection{Declarative (e.g. Logic) Programming}
Say we look at this same decision and action from a declarative programming
paradigm, and specifically implementing first order predicate logic.
\begin{equation} \label{eq:logic1}
  \large
  \begin{aligned}
    \exists\ obj_i \in &\{Food\ \wedge\ \forall obj_n \in Food, \text{where }
    n \neq i, \neg closer(obj_n,obj_i)\}
  \end{aligned}
\end{equation}

Equation \eqref{eq:logic1}, like the code above, describes whether or 
not to eat an object. If \eqref{eq:logic1} evaluates as $True$, then the agent 
will be told to eat the object. Now we have introduced other problems! For 
instance, we have the set $Food$ to initially populate \emph{and} parse 
through every time the agent considers an object. 

\subsubsection{As if it Wasn't Bad Enough Already...}
These two approaches have already proven to be ineffective in accomplishing
our goal, but then we must enforce a particularly harsh constraint: 
\textbf{there is no oracle in Flatworld}. That is, the agent must acquire
important knowledge on its own. Object distance and type are not simply
dictated to the agent.\footnote{Although later we shall reveal that we did
do this, and discuss why we could justify this.} Even though creating and
scanning the set $Food$ can be done in $O(n)$ time, our agent has no working
knowledge of where all the food objects are, or even how many exist
in Flatworld.


\subsection{Effective Theory}
It may not come as a surprise that we decided to approach this problem using
neural networks. We were perhaps a little harsh on our discussion of 
procedural programming. First of all, Flatland itself is implemented in C,
along with the agent. Secondly, we can still apply the mathematics involved
in neural networks with procedural languages like C. Rather than using a
rigid conditional structure, however, we programmed our agent's brains to
be able to adjust themselves based upon the situations the agent encounters.

\subsection{Our Implementation}
{\Large NEA: AM I COVERING ALL THE BASES?}
For all of our runs we used a shim\footnote{See section~\ref{sec:Ack}.} that 
sat between the original C code and our Python code. This served two purposes:
since we already had a functional object to describe neurons in Python we could
recycle it for this project; that and we only had to compile the C code once.
Given the number of minute changes we made to each incarnation of our brains,
avoiding recompiling every time allowed for immediate gratification (or 
disappointment) on our part.

 Our next goal was to create an agent who
would move blindly and devour any food/poison/placebo objects in its path.
As we improved its capabilities, the agent started to descriminate between 
helpful and detrimental objects, and ultimately it would only consume food 
objects when it was in need of food (i.e. it will not be a gluttonous agent).

\subsection{Brain Design}

\subsubsection{Brain 0}
We started from the most simplistic of models: an agent who sat in place until
it died. There is not much more to say about this model, other than it shunned
all of its inputs, and used none of its functions.

\begin{figure}
\begin{center}
  \includegraphics[scale=.3]{img/brain1.png}
  \caption{Active neurons in Brain 1}
  \label{fig:brain1}
\end{center}
\end{figure}

\subsubsection{Brain 1}

Our second brain (Fig. \ref{fig:brain1}) constantly activates the agent's 
eating actuator (i.e. the agent will try to eat anything with which it comes
in contact.

\subsubsection{Brain 2}

Brain 2 was the first neural network to take advantage of the agent's visual
cortex (Fig. \ref{fig:brain2}). Of the 31 eyelets available to the agent, it 
used only the data it received from it's center eyelet (eyelet 15).


\begin{figure}
\begin{center}
  \includegraphics[scale=.3]{img/brain2.png}
  \caption{Brain 2's visual neuron}
  \label{fig:brain2}
\end{center}
\end{figure}

\section{Brain Design} \label{sec:brain}
What follows is a series of descriptions of each of our brain models and what
features they added to the agent's functionality. In general, we wanted to 
start with a brain dead agent, then add to it movement, consumption, 
perception, and eventually judgment. Our figures in this section are used to 
illustrate the elements we have adjusted for each model. The full 
architecture of each model can be seen in appendix~\ref{ap:arch}.

\subsection{Brain 0}
We started from the most simplistic of models: an agent who sits in place until
it dies. It shuns all of its inputs, and uses none of its actuators.

\subsection{Brain 1}

Brain 1 (fig. \ref{fig:brain1}) moves in a straight line and constantly 
activates the agent's eating actuator (i.e. the agent will try to eat anything).

\begin{figure}
\begin{center}
  \includegraphics[scale=.3]{img/brain1.png}
  \caption{Active neurons in Brain 1}
  \label{fig:brain1}
\end{center}
\end{figure}

\subsection{Brain 2}

Brain 2 was the first neural network to take advantage of the agent's visual
cortex (fig. \ref{fig:brain2}). Of the 31 eyelets available to the agent, it 
used only the data it received from its center eyelet (eyelet 15). This
neuron instructs the agent to only use its eat actuator when it perceives
intense brightness. 

We caused this behavior by using a single neuron with a step function. 
The neuron was trained by having the agent try to eat at a variety of 
distances and observing the change in its charge. If the charge does not
change, then the brightness threshold is increased.

\begin{figure}
\begin{center}
  \includegraphics[scale=.3]{img/brain2.png}
  \caption{Brain 2's visual neuron}
  \label{fig:brain2}
\end{center}
\end{figure}

\subsection{Brain 3}

For Brain 3, we actually made two changes. The first was to the
eye neuron, which we changed from a step function to a sigmoid 
(fig. \ref{fig:brain3eye}):

\begin{equation} \label{eq:eyesig}
  \varphi(v) = \frac{\tanh(2v)}{10}
\end{equation}

We made this change because we wanted this model to focus on color. Whereas
our old step function classified \emph{all} objects as ``close enough'' vs. 
``not close enough,'' this new version could classify the color of an object by 
consuming it and comparing the output to the change in energy. Since this 
change 
can only be -0.1, 0, or 0.1, we selected a sigmoid function that would return 
a value within that range. Using a continuous function like equation 
\eqref{eq:eyesig} allows us to compare colors of objects.

\begin{figure}
\begin{center}
  \includegraphics[scale=.7]{img/brain3eye.png}
  \caption{Brain 3's visual neuron}
  \label{fig:brain3eye}
\end{center}
\end{figure}

Our second change was to employ the somatic sensors of the agent
(fig. \ref{fig:brain3touch}). If the agent senses contact on any of its edges, 
it will eat. By using these somatic sensors, we keep our agent from slovenly
gnashing its teeth at every time step.

\begin{figure}
\begin{center}
  \includegraphics[scale=.5]{img/brain3touch2.png}
  \caption{Brain 3's somatic neuron}
  \label{fig:brain3touch}
\end{center}
\end{figure}

\subsection{Brain 4}

Our mission for Brain 4 was to implement the eyelets into a
winner-take-all network (fig. \ref{fig:brain4}). Each eyelet input is fed 
into two neurons: one \textbf{step function} to determine green/not green, and 
another 
which produced the linear summation of each color's intensity. These outputs
were fed into product neurons, which subsequently output to a 31-element WTA 
network. All 
neurons in this network connect to a single neuron that controls the rotation 
of the agent, and this neuron has fixed weights coming into it. Each of these 
weights correspond to an angle, which is the direction that eyelet is looking. 
These angles are used to instruct the agent where to turn.

\begin{figure}
\begin{center}
  \includegraphics[scale=.52]{img/brain4.png}
  \caption{Brain 4's WTA eyelet network}
  \label{fig:brain4}
\end{center}
\end{figure}

\subsubsection{Waaaait a minute....}
We know what you're thinking. ``Step functions to determine color? But Brain 
3 introduced a sigmoid-based neuron!'' Indeed it is true. Furthermore, Brain
4 exerts no control over the eating actuator either. The reason for this is
that we, as the developers, simultaneously worked on Brains 3 and 4 separately.
Our initial thought was to fuse the two models together and toss the originals 
out. As you will see shortly, we did combine the two. However we kept Brains 
3 and 4 as a demonstration of a sort of evolutionary split in our agent's 
development. So without further ado...

\subsection{Brain 5}
Brain 5 applies the finer-grained sigmoid color neurons from 
fig. \ref{fig:brain3eye} to the augmented WTA network in 
fig. \ref{fig:brain4}. The object consumption is again determined by the 
somatic sensors from fig. \ref{fig:brain3touch}.

\subsection{Brain 6}
At this point in development, we wanted to refine the touch-and-munch nature
of our agent; why would it eat poison when it should know better? In order
to circumvent this greedy, potentially deadly tendency, we added a
color-based neuron to the agent's somatic-gastronomic system 
(fig. \ref{fig:brain6}). The somatic neuron will only fire if there is a
non-zero value for the ``mouth'' of the agent. Correspondingly, the
color-based neuron will only fire if an object in contact with it is green.
This is legal to assume, as the agent itself is aware of its own body. It will
never try to eat unless it has made contact with and object, and even then
will only do so if the object is green. This behavior is guaranteed by an
AND neuron, which serves as a gate between the touch and color neurons.

\begin{figure}
\begin{center}
  \includegraphics[scale=.7]{img/brain6.png}
  \caption{Brain 6's neural network for eating objects}
  \label{fig:brain6}
\end{center}
\end{figure}

\subsection{Brain 7}
\begin{figure}
\begin{center}
  \includegraphics[scale=.5]{img/brain7.png}
  \caption{Brain 7's movement network near food}
  \label{fig:brain7}
\end{center}
\end{figure}

Our seventh and final brain addresses the issue of rationing food supplies in
Flatworld. As previously stated, our agent has a charge/lifespan that is defined
as a real number between 1 and 0, with 1 being a full charge and 0 meaning 
death. Food will increase the agent's charge by \emph{at most} +0.1. Eating 
a food object with a charge of 0.97 brings the agent's charge up to 1, a net
gain of +0.03. In order to get the most out of food objects, our agent
should never eat when it's not hungry. Fig. \ref{fig:brain7} describes our
network to help accomplish this goal.

 Most of the time the agent has the standard propulsion methods as it did in 
other models, fig. \ref{fig:brain7} is put into use under the special 
circumstances that our agent is very close to a food object, but is not hungry.
A constant value describes the forward movement rate (0.25 in this case). At 
every timestep the agent queries its internal charge. While its charge is 
$> 0.9$, the logical NOT will suppress the AND neuron, giving the 
forward/backward 
actuator a value of $0.25 \times 0 = 0$. Provided the object in question is 
close enough (described by the intensity neuron), when the agent's energy dips
below 0.9 the NOT neuron fires, the AND neuron is satisfied, and it fires a
non-zero value to the product neuron.

\section{Results} \label{sec:Results}
\begin{itemize}
  \item
    JUST THE OBJECTIVE FACTS, no interpretation or discussion
  \item
    what exactly did you do
  \item
    what were the inputs/outputs/parameters
  \item
    what computer hardware
  \item
    what software
  \item
    what were the results
  \item
    tables/figures/plots
  \item
    error analysis
\end{itemize}

\documentclass[a4paper,11pt]{article}
\begin{document}


\section{Discussion} \label{sec:Discussion}

\begin{itemize}
   \item
     your opinion of the study.
   \item
     your opinion of the results
   \item
     interpretations
   \item
     comments
   \item
     analysis
   \item
     what was right/wrong
   \item
     improvements
   \item
     connections back to the literature
\end{itemize}
\end{document}
\section{Summary} \label{sec:Summary}
\begin{itemize}
  \item
    the topic
  \item
    the problem
  \item
    other's solutions
  \item        
    your solution
  \item
    what was right/wrong about it
  \item
    future work
  \item
    conclusion 
\end{itemize}

\section{Acknowledgements} \label{sec:Ack}
The authors would like to thank Nasser Salim for his C-to-Python shim that
we used in our brain implementations, Tom Caudell for his Flatworld
environment, and the UNM's Department of Computer Science for all they have
done during our time here.

\appendix

%Import our appendices here:
\section{Model Architecture} \label{ap:arch}

\begin{figure}
\begin{center}
  \includegraphics[scale=.3]{img/arch1.png}
\end{center}
\end{figure}

\begin{figure}
\begin{center}
  \includegraphics[scale=.45]{img/arch2.png}
\end{center}
\end{figure}

\input{Ap2.tex}


\end{sloppypar}

\begin{thebibliography}{9}

\bibitem{Anders}
  Anders, Torsten, and Eduardo R. Miranda. ``Constraint Programming Systems for Modeling Music Theories and Composition.'' \underline{ACM Computing Surveys} 43.4 (2011): 1:38.

\end{thebibliography}

\end{document}


